% $Id: commands.tex 80 2009-04-09 11:06:26Z koppor $
%
%wird fuer Tabellen benoetigt (z.B. >{centering\RBS}p{2.5cm} erzeugt einen zentrierten 2,5cm breiten Absatz in einer Tabelle
\newcommand{\RBS}{\let\\=\tabularnewline}

%Normalerweise werden Texte mittels ``Satz'' beschrieben.
%Dann werden auch die richtigen Anfuehrungszeichen verwendet.
%Manchmal klappt es doch nicht. Und statt \glqq Test\grqq{} ... wird
%hier das Kommando \gq definiert, dass sich \gq{so} benutzt.
% siehe reader für richtige variante
\newcommand{\gq}[1]{\glqq{}#1\grqq{}\xspace}

%% typoraphisch richtige Abkürzungen
\newcommand{\zB}[0]{z.\,B.\xspace}
\newcommand{\bzw}[0]{bzw. \xspace}
\newcommand{\usw}[0]{usw. \xspace}

%from hmks makros.tex - \indexify
\newcommand{\toindex}[1]{\index{#1}#1}
%
\newcommand{\dotcup}{\ensuremath{\,\mathaccent\cdot\cup\,}} %Tipp aus The Comprehensive LaTeX Symbol List
%
%Anstatt $|x|$ $\abs{x}$ verwenden. Die Betragsstriche skalieren automatisch, falls "x" etwas groessers sein sollte...
\newcommand{\abs}[1]{\left\lvert#1\right\rvert}
%
%fuer Zitate
\newcommand{\citeS}[2]{\cite[S.~#1]{#2}}
\newcommand{\citeSf}[2]{\cite[S.~#1\,f.]{#2}}
\newcommand{\citeSff}[2]{\cite[S.~#1\,ff.]{#2}}
\newcommand{\vgl}{vgl.\ }
\newcommand{\Vgl}{Vgl.\ }
%
\newcommand{\commentchar}{\ensuremath{/\mkern-4mu/}}
\algrenewcommand{\algorithmiccomment}[1]{\hfill $\commentchar$ #1}

% Seitengroessen - Gegen Schusterjungen und Hurenkinder...
\newcommand{\largepage}{\enlargethispage{\baselineskip}}
\newcommand{\shortpage}{\enlargethispage{-\baselineskip}}
