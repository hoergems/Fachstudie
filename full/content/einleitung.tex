%Folgende Zeile aktivieren und als SVN property "svn:keywords" auf "Id" setzen, um SVN Versionsinformationen im Dokument zu erhalten
%\svnInfo $Id: einleitung.tex 60 2012-01-26 15:56:06Z koppor $ 

\chapter{Einleitung}
In diesem Kapitel steht die Einleitung zu dieser Diplomarbeit. Sie soll nur als Beispiel dienen und hat nichts mit dem Buch \cite{WSPA} zu tun. Nun viel Erfolg bei der Arbeit!

\section*{Gliederung}
Die Arbeit ist in folgender Weise gegliedert:
\begin{description}
\item[Kapitel~\ref{chap:k2} -- \nameref{chap:k2}:] Hier werden werden die Grundlagen dieser Arbeit beschrieben.
\item[Kapitel~\ref{chap:zusfas} -- \nameref{chap:zusfas}] fasst die Ergebnisse der Arbeit zusammen und stellt Anknüpfungspunkte vor.

Nur das es vorkommt \cite{WSPA}.
\end{description}
