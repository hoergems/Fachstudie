%Folgende Zeile aktivieren und als SVN property "svn:keywords" auf "Id" setzen, um SVN Versionsinformationen im Dokument zu erhalten
%\svnInfo $Id: einleitung.tex 60 2012-01-26 15:56:06Z koppor $ 

\chapter{Einleitung}
Das Forschungsthema des Aufgabenstellers befasst sich mit der computergestützten Analyse von aufgebauten, passiven und planaren Hochfrequenzschaltungen. Dies geschieht mit Hilfe von höchst präzisen 3D Modellen, die auf Basis von Messungen mit einem Computertomographen erstellt werden. Da die Ausgaben des digitalen Röntgendetektors in Form von diskreten Voxeldaten vorliegen, werden von den derzeit eingesetzten Algorithmen zur Objekterzeugung nur Objekte mit rauhen und kantigen Oberflächen erzeugt. Leider führt dies bei der Analyse zu Nachteilen bei der Bestimmung der elektrischen Hochfrequenzeigenschaften.

\section*{Gliederung}
Die Arbeit ist in folgender Weise gegliedert:
\begin{description}
\item[Kapitel~\ref{chap:k2} -- \nameref{chap:k2}:] Hier werden werden die Grundlagen dieser Arbeit beschrieben.
\item[Kapitel~\ref{chap:zusfas} -- \nameref{chap:zusfas}] fasst die Ergebnisse der Arbeit zusammen und stellt Anknüpfungspunkte vor.

Nur das es vorkommt \cite{WSPA}.
\end{description}
